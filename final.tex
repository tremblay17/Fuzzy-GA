\documentclass{IEEEtran}

\usepackage[english]{babel}
\usepackage[letterpaper,top=2cm,bottom=2cm,left=3cm,right=3cm,marginparwidth=1.75cm]{geometry}
\usepackage{amsmath}
\usepackage{graphicx}
\usepackage[colorlinks=true, allcolors=blue]{hyperref}

\title{Fuzzy Logic \& Genetic Algorithms in Agricultural Drones}
\author{
    \IEEEauthorblockN{Dylan Tribble}
    \and
    \IEEEauthorblockN{Jacob Adams}
}

\begin{document}
\maketitle
\begin{abstract}
Over recent years, fuzzy logic, with the help of drones and UAVs, has made continuous advancements in agriculture to aid in crop monitoring, resource management, and risk assessment. While the combination of the two has improved the field of agriculture, the addition of genetic programming can further UAV potential by offering decision support in navigation and obstacle avoidance, automating data collection on crops, and aiding in analysis and prediction models. This research aims to find how these areas can come together in a way that not only saves time and money in man hours but also maximize overall yield by minimizing resources and combating crop diseases. By developing an expert system that is intelligent and adaptive, drones will be equipped to make decisions while being context-aware of the environment, variables related to a particular crop's yield and objects that they measure. The objective is to have a drone that can recognize problems for the given area of crops, navigate complex and dynamic environments, and perform data collection all while progressively learning. This will allow farmers to better maintain their crops by knowing not only what to adjust regarding the resources used to increase yield but also their amounts. To complete this, a literature review will be executed to determine the most effective methods for determining decision-making with fuzzy logic and genetic programming. Jacob Adams will focus on the areas of flight path planning and obstacle avoidance while Dylan Tribble will look at the expert systems and implementations used to create the capability of recognizing changes in plant health and payload delivery. 
\end{abstract}

\section{Introduction}
On any given day, the world adds approximately 200 thousand people to the population, with the United Nations projecting the global population to surpass 9 billion by 2050 \cite{key9} \cite{United_Nations}.
Critically, food production must match the rapid growth of the world's population, with many farmers adding Artificial Intelligence (AI) to their arsenal of tools to meet this goal. Traditional, manual sampling
can be destructive to crops and is prone to bias and human error. While AI, when combined with Unmanned Air Vehicles (UAVs) also known as drones, can play a vital role in increasing yield, along with environmental
sustainability through the analysis of key variables such as canopy cover, crop height, nitrogen content, and water stress \cite{key7}. While there exists a variety of methods to automate drone usage on farms,
the first hurdle to overcome is path planning when orchestrating the routes of a multitude of drones. Path planning for a team of drones can either be handled through a centralized controller, or with each UAV making decisions
by itself.

%In order to increase food production, farmers are turning to Unmanned Air Vehicles (UAVs), colloquially referred to as drones. UAVs are being tested as a modern approach to crop analysis and in the regulation
%and application of fertilizers, herbicides, and pesticides. 
% \begin{itemize}
% \item Provide an overview of the importance of precision agriculture and the role of drones in modern farming.
% \item Introduce the need for advanced technologies like fuzzy logic and genetic algorithms in optimizing agricultural drone performance.
% \item Define the scope and objectives of your review.
% \end{itemize}

\section{Precision Agriculture}
Precision agriculture (PA) is the practice of collecting and analyzing data to automate the farming process through removing human bias and reducing destructive manual testing practices. PA started decades ago
in the form of unmanned aircraft and satellite imaging. Both of which had questionable accuracy, along with being heavily dependent on ideal weather scenarios \cite{key9}. Satellite data, which is used
to monitor broadacre farming, provides detail on the scale of meters, while drones offer the potential for centimeter resolution \cite{key8}. The push for PA is being driven by two main factors: waste reduction in the form
of time, resources, and yield which translates into cost savings, and environmental protection \cite{key9}.

    % \begin{itemize}
        % \item Briefly discuss the evolution of precision agriculture and the role of technology in improving farming practices.
    % \end{itemize}
\subsection{Agricultural Drones}
Drones applied in agriculture are seeing wide spread research and use for imagery, in order to track the health and progress of plants throughout the season.
Data being tracked can include water stress, soil type, pests, weeds, and fungal infestations, depending on the farmer’s needs \cite{key9}. For plant monitoring,
drones typically rely on visible light (RGB), multispectral, hyperspectral, thermal, or any combination of these \cite{key7} \cite{key8} \cite{key9}. Further,
drones are being developed to create augmented reality visualizations of crops, and using RGB imagery to monitor soil pH levels along with other key nutrients
for the crops \cite{key7}.

    % \begin{itemize}
    %     \item Discuss the role of drones.
    %     \item Highlight the key features and capabilities of agricultural drones.
    % \end{itemize}
\subsection{Hybrid Systems}
    \subsubsection{Fuzzy Logic}
        \begin{itemize}
            \item Define fuzzy logic and its applications in the context of agricultural drones.
            \item Summarize recent research on the use of fuzzy logic for decision-making, control, and navigation in agricultural drone systems.
            \item Discuss the advantages and limitations of applying fuzzy logic in this domain.
        \end{itemize}
    \subsubsection{Genetic Algorithms}
        \begin{itemize}
            \item Introduce genetic algorithms and their relevance to optimization problems in agricultural drone applications.
            \item Review recent studies and implementations that utilize genetic algorithms for route planning, resource optimization, or other tasks in agricultural drone systems.
            \item Evaluate the effectiveness of genetic algorithms in addressing specific challenges in agriculture.
        \end{itemize}

\section{Integration Techniques}
    \begin{itemize}
        \item     Explore research that combines fuzzy logic and genetic algorithms to enhance the performance of agricultural drone systems.
        \item Discuss how the synergy of these two techniques addresses specific challenges and optimizes various aspects of drone operations in agriculture.
    \end{itemize}
\section{Approaches for Hybrid Systems in UAVs}
    \begin{itemize}
        \item Provide real-world examples or case studies that demonstrate the successful implementation of fuzzy logic and genetic algorithms in agricultural drone technology.
        \item Discuss the outcomes, benefits, and lessons learned from these case studies.
    \end{itemize}

\section{Designing a Hybrid System for UAVs}
    \begin{itemize}
        \item 
    \end{itemize}

\section{Challenges and Outlook}
    \begin{itemize}
        \item Identify current challenges and limitations in the integration of fuzzy logic and genetic algorithms in agricultural drones.
        \item Discuss potential future directions for research and development in this field, including emerging technologies or methodologies. \cite{key4}
    \end{itemize}

\section{Conclusion}
    \begin{itemize}
        \item Summarize the key findings and insights from your review.
        \item Emphasize the significance of using fuzzy logic and genetic algorithms in advancing agricultural drone technology.
        \item Provide recommendations for future research in this area.
    \end{itemize}

\bibliographystyle{IEEEtran}
\bibliography{references}

\end{document}

